\documentclass[11pt, a4paper]{article}

% --- Packages ---
\usepackage[utf8]{inputenc}
\usepackage[T1]{fontenc}
\usepackage{mathpazo} % Professional serif font (Palatino)
\usepackage[english]{babel}
\usepackage{geometry}
\geometry{margin=1in}
\usepackage{graphicx}
\usepackage{booktabs}
\usepackage{hyperref}
\usepackage{listings}
\usepackage{xcolor}

% --- Code Snippet Styling ---
\lstset{
    backgroundcolor=\color{gray!5},
    basicstyle=\ttfamily\small,
    breaklines=true,
    frame=single,
    language=Python,
    keywordstyle=\color{blue},
    commentstyle=\color{gray},
    stringstyle=\color{orange}
}

% --- Document Metadata ---
\title{Pharmaceutical Market Analysis: Drug Characteristics and Prescription Trends}
\author{Ahmed ElShafiey}
\date{\today}

\begin{document}

\maketitle

\begin{abstract}
This report analyzes pharmaceutical data from \url{kaggle.com}, 
examining drug classifications (Rx vs. OTC), pregnancy safety 
categories, and patient usage patterns across medical conditions. 
Pain is the most common condition treated, followed by cold 
symptoms and acne, with pain relief showing the highest activity. 
Most painkillers are OTC, though potent options require 
prescriptions, and over 50\% fall under pregnancy category C. 
Cold medications are predominantly OTC, with limited safety 
data for pregnancy. High-prevalence conditions like Hypertension 
and Rheumatoid Arthritis involve more restricted Category D 
and X drugs, while lower-prevalence conditions rely more on 
Category B treatments. Overall, prescription medications dominate 
across conditions, though some, like Hayfever and Osteoarthritis, 
are readily available OTC, whereas complex conditions such as 
Diabetes and Pneumonia are managed primarily with prescriptions.

\end{abstract}

\section{Introduction}
The dataset analyzed in the notebook \texttt{analysis.ipynb} 
contains comprehensive information regarding drug names, 
medical conditions, and regulatory classifications. 
The analysis aims to clean the raw data and visualize 
the market landscape for both prescription (Rx) 
and over-the-counter (OTC) medications. Utilizing 
such raw data opens our eyes to hidden patterns 
and trends in drug prescriptions for many common 
medical conditions, helping to improve drug 
effectiveness and prevent drug-induced adverse effects.

\section{Data Description}
This dataset obtained from Kaggle: 
\url{https://www.kaggle.com/datasets/jithinanievarghese/drugs-related-to-common-treatments}.
The dataset includes several critical features used for pharmaceutical evaluation:
\begin{itemize}
    \item \textbf{Activity:} A metric of site visitor engagement relative to other medications.
    \item \textbf{Rx/OTC:} Classification of whether a drug requires a prescription.
    \item \textbf{Pregnancy Category:} Safety ratings ranging from A (safe) to X (contraindicated).
    \item \textbf{CSA:} Controlled Substances Act schedule (potential for abuse).
    \item \textbf{Rating:} User-reported effectiveness on a scale of 1 to 10.
\end{itemize}

\section{Data Preprocessing}
\subsection{Handling Missing Values}
Preliminary analysis revealed significant missing data in specific columns:
\begin{itemize}
    \item \textbf{Alcohol, Rating, and No\_of\_reviews:} Excluded due to nearly 50\% missingness.
    \item \textbf{Pregnancy Category:} 6.28\% missing values were imputed using the mode (\textbf{Category C}).
    \item \textbf{Rx/OTC:} Standardized by treating ``Rx/OTC'' hybrids as ``OTC'' for consumer accessibility analysis.
\end{itemize}

\subsection{Standardization Code}
The following logic was applied to ensure data integrity:
\begin{lstlisting}[language=Python]
df['pregnancy_category'] = df['pregnancy_category'].fillna('C')
df['rx_otc'] = df['rx_otc'].replace('Rx/OTC', 'OTC')
df['rx_otc'] = df['rx_otc'].fillna('Rx')
\end{lstlisting}

\begin{lstlisting}[language=Python]
cols_to_drop = ['alcohol', 'rating', 'no_of_reviews']
cols_existing = [col for col in cols_to_drop if col in df.columns]
df.drop(columns=cols_existing, inplace=True)
\end{lstlisting}

\section{Exploratory Data Analysis}
The processed dataset contains \textbf{3,959} records. 
The distribution of medical conditions shows that 
\textbf{Pain} (393 drugs), \textbf{Colds \& Flu} 
(246 drugs), and \textbf{Acne} (238 drugs) are 
the most highly represented categories.

\subsection{Prescription vs. OTC Distribution}
The analysis confirms that a majority of the 
medications in the dataset (\textbf{2,702}) 
are prescription-only (Rx), compared to \textbf{1,257} 
available over-the-counter.

\subsection{Analgesics Ranked by User Activity}
Activity reflects relative user engagement and usage 
trends based on recent site interactions, rather 
than clinical potency. The observed 
rankings—particularly the unexpectedly high 
activity of Ibuprofen compared to more 
potent analgesics—suggest that this metric 
primarily captures frequency of use, prescription 
volume, or over-the-counter accessibility.Moreover, 
the pronounced gap between Acetaminophen 
and other analgesics indicates that factors beyond 
pharmacological strength, such as population-wide 
usage patterns and data aggregation methods, 
significantly influence the metric. 
These findings imply that Activity should be 
interpreted as a behavioral and utilization 
indicator, rather than a direct measure of 
therapeutic potency.

\subsection{Analgesics OTC drugs compared to RX drugs}
For analgesic prescribing patterns, OTC analgesics 
account for \textbf{236} medications, while 
POM analgesics total \textbf{157}

\subsection{Distribution of Pain Drugs by Pregnancy Category}
Over 50\% of the analyzed painkillers fall under 
pregnancy category C, with no drugs in category A 
and only 13.5\% classified as category B. 
This distribution emphasizes that painkillers 
should be prescribed during pregnancy only as a 
last resort, following careful evaluation. 
Additionally, approximately 30\% of the drugs 
are categorized as N (not yet assessed for pregnancy 
safety), representing the second-largest group. 
The substantial proportion of unassessed medications 
underscores the urgent need for further evaluation 
to ensure the safety of painkillers during pregnancy.

\subsection{Cold \& Flu in each pregnancy category}
Most cold and flu medications fall into Category N 
(unclassified) or Category C, indicating limited 
or uncertain data regarding pregnancy safety. In 
contrast, Category B drugs are relatively few, and 
Category D drugs are rare, reflecting their 
restricted use due to known fetal risks. 
Overall, this underscores the limited availability 
of clearly pregnancy-safe options and highlights 
the importance of cautious prescribing and patient 
counseling during pregnancy.

\subsection{Other medical conditions – drug distribution by pregnancy category}
The distribution of drugs by pregnancy category across 
various medical conditions shows that Acne and Hayfever 
have the highest total drug counts. 
Notably, high-count conditions such as Hypertension 
and Rheumatoid Arthritis exhibit a significant 
presence of Category D and X drugs, reflecting 
more restricted use, whereas lower-count conditions 
like UTI and Pneumonia rely more on Category B 
treatments. Overall, the chart indicates that, 
for most common conditions, available pharmaceutical 
options are predominantly Category C, highlighting a 
potential need for further clinical investigation 
into safer alternatives during pregnancy.

\subsection{Other medical conditions – OTC vs Rx distribution}
There is a strong reliance on prescription (Rx) 
medications across nearly all conditions. 
While Hayfever and Osteoarthritis exhibit 
notable over-the-counter (OTC) accessibility, 
more complex conditions such as Diabetes and 
Pneumonia are managed almost exclusively 
through regulated prescriptions.

\section{Visual Analysis}
The notebook utilizes a custom visual palette for professional presentation:
\begin{itemize}
    \item \textbf{Primary Blue (\#1f4e79):} Used for prescription data.
    \item \textbf{Light Blue (\#9dc3e6):} Used for OTC data.
\end{itemize}

\section{Results}
The exploratory analysis of the dataset, comprising 3,959 
drug records, revealed that Pain (393 drugs), 
Colds \& Flu (246 drugs), and Acne (238 drugs) 
are the most represented medical conditions. 
Across nearly all conditions, there is a strong reliance 
on prescription (Rx) medications, with 2,702 drugs 
requiring prescriptions compared to 1,257 available 
over-the-counter (OTC). OTC accessibility is 
notable for common conditions such as Hayfever and 
Osteoarthritis, whereas complex conditions like 
Diabetes and Pneumonia are managed almost 
exclusively with regulated prescriptions. 
Analgesic user activity, reflecting engagement 
rather than therapeutic potency, shows that 
Ibuprofen exhibits unexpectedly high usage, 
likely influenced by factors such as frequency 
of use, prescription volume, and OTC 
availability, while Acetaminophen demonstrates 
a pronounced activity gap compared to other 
analgesics. For analgesic prescribing patterns 
specifically, OTC options slightly outnumber 
prescription-only drugs (236 vs. 157), 
highlighting consumer preference for 
accessible pain relief.

The distribution of drugs by pregnancy category 
underscores limited safety data for many common 
medications. Over 50\% of painkillers fall 
under Category C, only 13.5\% are in Category B, 
none in Category A, and approximately 30\% 
remain unassessed (Category N). Cold and flu 
drugs similarly cluster in Categories N and C, 
with few Category B options and rare 
Category D drugs, reflecting restricted use 
due to fetal risk. Across other medical 
conditions, high-count categories such as 
Hypertension and Rheumatoid Arthritis 
include more Category D and X drugs, while 
lower-count conditions like UTI and Pneumonia 
rely more on Category B treatments. Overall, 
Category C dominates most conditions, 
highlighting gaps in clearly pregnancy-safe 
options and emphasizing the importance of 
cautious prescribing, patient counseling, 
and further clinical evaluation. 
The findings also indicate that observed 
trends in drug activity and OTC accessibility 
are influenced not just by pharmacological 
properties but by usage patterns, 
prescription practices, and population-wide behaviors.

\section{Conclusion}
The analysis demonstrates that while the market for common ailments (Pain and Colds) is diverse, the majority of therapeutic options remain under regulatory prescription control. Future analysis could incorporate the excluded rating data if a more complete longitudinal dataset becomes available.

\section*{Appendix: Technical Implementation}
The analysis was performed using \texttt{pandas}, \texttt{seaborn}, and \texttt{matplotlib}. Global font settings were forced to \textit{serif} to match professional reporting standards.

\end{document}